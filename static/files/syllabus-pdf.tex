\documentclass{article}
\usepackage{booktabs, graphicx, hyperref, fontspec}
\usepackage{sectsty}
\allsectionsfont{\sffamily}
\usepackage[margin=1in]{geometry}
\hypersetup{
  colorlinks = true,
  urlcolor = cyan,
 }
 \providecommand{\tightlist}{%
  \setlength{\itemsep}{0pt}\setlength{\parskip}{0pt}}
\newcommand*{\authorfont}{\fontfamily{phv}\selectfont}
\usepackage[]{Fira Sans}

\begin{document}

\sffamily

\centerline{\Huge Econometrics}

\vspace{3 mm}

\centerline{\large Dr.~Ryan Safner}
\vspace{2 mm}
\centerline{\large \href{http://metricsF20.classes.ryansafner.com}{metricsF20.classes.ryansafner.com}}

\vspace{5 mm}

\begin{tabular}{@{}p{3.5in}p{3.5in}}           
\textbf{Course}: ECON 480 Fall 2020  & \textbf{Email:}  \href{mailto:safner@hood.edu}{\nolinkurl{safner@hood.edu}}\\
\textbf{Room}: ON ZOOM & \textbf{Office:}  Rosenstock 118\\
\textbf{Meets}: TuTh 11:30 A.M.--12:45 P.M. & \textbf{Hours:} TuTh 3:30--5:00PM ON ZOOM\\ 
\end{tabular}

\vspace{5 mm}

\hrule


\begin{quote}
``There are three kinds of lies: lies, damned lies, and statistics.'' --
Benjamin Disraeli
\end{quote}

\textbf{Econometrics} is the application of statistical tools to
quantify and measure economic relationships in the real world. It uses
real data to test economic hypotheses, quantitatively estimate causal
relationships between economic variables, and to make forecasts of
future events. The primary tool that economists use for empirical
analysis is ordinary least squares (OLS) linear regression, so the
majority of this course will focus on understanding, applying, and
extending OLS regressions.

I assume you have \emph{some} working knowledge of economics at the
intermediate level and some basic statistical tools.\footnote{The formal
  prerequisites for this course are \textbf{ECON 205} and \textbf{ECON
  206}; \textbf{ECMG 212} or \textbf{MATH 112}; and \textbf{ECON 305} or
  \textbf{ECON 306}} We will do some basic review of some necessary
statistics and probability at the beginning until everyone is
comfortable, before jumping right into regressions.

\hypertarget{hybrid-course-format}{%
\section*{Hybrid Course Format}\label{hybrid-course-format}}
\addcontentsline{toc}{section}{Hybrid Course Format}

This course is taught in a \textbf{hybrid} format, providing a mixture
of regular synchronous activity where we all can interact in real time,
with asynchronous material, which can be done remotely at your own pace.

\textbf{I will be holding all synchronous class sessions remotely (for
reasons I will make clear to you by the first day) on Zoom.} You can
attend these sessions on your computer or device from your dormitory or
home, and a classroom is available for you to use (socially-distanced,
and in masks), but I will not be in the classroom.

During the synchronous, scheduled times for the course (Monday/Wednesday
2:00 P.M.-3:15 P.M.), I will lecture on the material, hold in-class
discussions, and answer questions in real time \emph{on Zoom.}
Attendance to the live portion is strongly encouraged, but not required.

\textbf{Lecture slides, videos, and other synchronous materials will be
posted online by the end of the day in which the live session occurs.}

Assignments will always be submitted \emph{online} and due at regular
times (typically 11:59 PM Sunday) so that students unable to join in the
live sessions can complete them asynchronously.

Students are strongly encouraged to join the course
\href{https://hoodcollegeeconomics.slack.com}{Slack channel} to maintain
an active channel of communication, ask questions, and to build our
course community together. Official course-related announcements will
always come via Blackboard announcement and automatically sent to your
Hood email accounts.

\hypertarget{learning-in-a-time-of-coronavirus}{%
\subsection*{Learning in a Time of
Coronavirus}\label{learning-in-a-time-of-coronavirus}}
\addcontentsline{toc}{subsection}{Learning in a Time of Coronavirus}

Everything is awful right now. None of us signed up for this. None of us
are really okay, \textbf{we're all just pretending for everyone else.}

Many of you may be dealing with hardships at home and at work, and are
generally juggling many more problems than usual. Everyone's future
plans have been completely put on hold or cancelled to a large degree.
We all miss the sense of normalcy and human sense of community from
being isolated for so long.

For this unique semester, we are going to prioritize supporting each
other as human beings during this crazy era, and use simple, accessible
solutions that make sense for the most people, and above all, to be
flexible. I have designed the course to maintain some common structure
but be flexible to your varied needs. Please see the
\protect\hyperlink{policies-and-expectations}{policies and expectations
below}. I hope you use this course as an opportunity to escape the
boredom and insanity of social isolation, and to help keep interest in
understanding the world around us.

If you tell me you're having trouble, I will do whatever I can to help,
and not judge you or think less of you. I hope you will extend me the
same courtesy.

\hypertarget{course-objectives}{%
\section*{Course objectives}\label{course-objectives}}
\addcontentsline{toc}{section}{Course objectives}

\textbf{By the end of this course,} you will:

\begin{enumerate}
\def\labelenumi{\arabic{enumi}.}
\item
  apply the models of microeconomics (constrained optimization and
  equilibrium) towards explaining real world behavior of individuals,
  firms, and governments
\item
  explore the effects of economic and political processes on market
  performance (competition, market prices, profits and losses, property
  rights, entrepreneurship, market power, market failures, public
  policy, government failures)
\item
  apply the economic way of thinking to real world issues in writing
\item
  understand how to evaluate statistical and empirical claims;
\item
  use the fundamental models of causal inference and research design;
\item
  gather, analyze, and communicate with real data in R.
\end{enumerate}

I am less concerned with forcing you to memorize and recite proofs of
statistical estimator properties, and more concerned with the
development of your intuitions and the ability to think critically as an
empirical social scientist---although this will require you to
demonstrate proficiency with some intermediate statistical and
mathematical tools.

Given these objectives, this course fulfills all three of the learning
outcomes for
\href{https://www.hood.edu/academics/departments/george-b-delaplaine-jr-school-business/student-learning-outcomes}{the
George B. Delaplaine, Jr.~School of Business} Economics B.A. program:

\begin{itemize}
\tightlist
\item
  Use quantitative tools and techniques in the preparation,
  interpretation, analysis and presentation of data and information for
  problem solving and decision making {[}\ldots{]}
\item
  Apply economic reasoning and models to understand and analyze problems
  of public policy {[}\ldots{]}
\item
  Demonstrate effective oral and written communications skills for
  personal and professional success{[}\ldots{]}
\end{itemize}

\textbf{Fair warning: Econometrics is hard.} \emph{It will be one of the
hardest economics courses that you will take, primarily due to the
mathematical content.} I will do my best to make this class intuitive
and helpful, if not interesting. If at any point you find yourself
struggling in this course for any reason, please come see me. Do not
suffer in silence! Coming to see me for help does not diminish my view
of you, in fact I will hold you in \emph{higher} regard for
understanding your own needs and taking charge of your own learning.
There are also a some fantastic resources on campus, such as the
\href{http://www.hood.edu/campus-services/academic-services/index.html}{Center
for Academic Achievement and Retention (CAAR)} and the
\href{http://www.hood.edu/library/\%7D\%7BBeneficial-Hodson\%20Library}{Beneficial-Hodson
Library}.

See my
\href{http://metricsF20.classes.ryansafner.com/reference\#tips}{tips for
success in this course}.

\hypertarget{required-course-materials}{%
\section*{Required Course materials}\label{required-course-materials}}
\addcontentsline{toc}{section}{Required Course materials}

This course requires regular online internet access. If you know you
will be unable to access the internet regularly, please let me know and
we can make arrangements.

You can find all course materials at my \textbf{dedicated website} for
this course:
\href{https://metricsF20.classes.ryansafner.com}{metricsF20.classes.ryansafner.com}.
Links to the website are posted on our Blackboard course page. Please
familiarize yourself with the website, see that it contains this
\href{https://metricsF20.classes.ryansafner.com/syllabus/}{syllabus},
guides for your
\href{https://metricsF20.classes.ryansafner.com/reference/}{reference},
and our
\href{https://metricsF20.classes.ryansafner.com/schedule/}{schedule}. On
the schedule page, you can find each module with its own class page
(\textbf{start there!}) along with all related readings, lecture slides,
practice problems, and assignments.

My lecture slides will be shared with you, and serve as your primary
resource, but our main ``textbook'' below is \textbf{recommended} as the
next best resource and will be available from the campus bookstore. I
will discuss more about textbooks and materials in the first module.

\hypertarget{books}{%
\subsection*{Books}\label{books}}
\addcontentsline{toc}{subsection}{Books}

The following book is \textbf{required}\footnote{You are not
  \emph{obligated} to buy it, I just \textbf{strongly recommend} it in
  the sense that you will still have access to all data and assignments
  without possessing the book. But this is a course where you really
  will want to understand the derivations or get additional context
  beyond just my slides\ldots{}} and will be available from the campus
bookstore.

\begin{itemize}
\tightlist
\item
  Bailey, Michael A, 2019, \emph{Real Econometrics}, New York: Oxford
  University Press, 2\textsuperscript{nd} ed.
\end{itemize}

You are welcome to purchase the book by other means (e.g.~Amazon,
half.com, etc). I have no financial stake in requiring you to purchase
this book. The (cheaper) 1st edition is sufficient, but makes
significantly less use of \texttt{R} (in favor of \texttt{STATA}).

The following two books are \textbf{recommended}, and are free
online\footnote{You can purchase a hard copy of the first one if you
  really want.}:

\begin{itemize}
\tightlist
\item
  Grolemund, Garrett and Hadley Wickham,
  \href{https://r4ds.had.co.nz/}{\emph{R For Data Science}}
\item
  Ismay, Chester and Albert Y Kim,
  \href{https://moderndive.com/}{\emph{Modern Dive: Statistical
  Inference Via Data Science}}
\end{itemize}

The first book is the number one resource for using \texttt{R} and
\texttt{tidyverse}, and is written for beginners. I still look at it
\emph{frequently.} The second is another great reference for using
\texttt{tidyverse} in the context of basic statistics.

\hypertarget{software}{%
\subsection*{Software}\label{software}}
\addcontentsline{toc}{subsection}{Software}

You are \textbf{strongly recommended} to download copies of
\href{https://www.r-project.org/}{\texttt{R}} and
\href{http://www.rstudio.com}{\texttt{R\ Studio}} on your own computers.
These software packages are available on all computers in the trading
room, and you will have access to them during the week to work on
assignments.

We will also have a shared class workspace in
\href{http://rstudio.cloud/}{RStudio.cloud} that runs a full instance of
R Studio in your web browser (so no need to install anything!) will let
you access files and assignments.

\hypertarget{articles}{%
\subsection*{Articles}\label{articles}}
\addcontentsline{toc}{subsection}{Articles}

Throughout the course, I will post both required and supplemental
(non-required) readings that enrich your understanding for each topic.
Check Blackboard \emph{frequently} for announcements and updates to
assignments, readings, and grades.

\hypertarget{assignments-and-grades}{%
\section*{Assignments and Grades}\label{assignments-and-grades}}
\addcontentsline{toc}{section}{Assignments and Grades}

Your final course grade is the weighted average of the following
assignments. You can find general descriptions for all the assignments
on the
\href{http://metricsF20.classes.ryansafner.com/assignments/}{assignments
page} and more specific information and examples on each assignment's
page on the
\href{http://metricsF20.classes.ryansafner.com/schedule/}{schedule
page}.

\begin{center}

\begin{tabular}{lll}
\toprule
 & Assignment & Percent\\
\midrule
1 & Research Project & 30\%\\
n & Problem sets (Average) & 25\%\\
1 & Midterm & 20\%\\
1 & Final & 25\%\\
\bottomrule
\end{tabular}
\end{center}

Each assignment is graded on a 100 point scale. Letter-grade equivalents
are based on the following scale:

\begin{center}

\begin{tabular}{llll}
\toprule
Grade & Range & Grade1 & Range1\\
\midrule
A & 93–100\% & C & 73–76\%\\
A− & 90–92\% & C− & 70–72\%\\
B+ & 87–89\% & D+ & 67–69\%\\
B & 83–86\% & D & 63–66\%\\
B− & 80–82\% & D− & 60–62\%\\
\addlinespace
C+ & 77–79\% & F & < 60\%\\
\bottomrule
\end{tabular}
\end{center}

See also my
\href{https://ryansafner.shinyapps.io/480_grade_calculator/}{
\texttt{Grade\ Calculator}} app where you can calculate your overall
grade using existing assignment grades and forecast ``what if''
scenarios.

These grades are firm cutoffs, but I do of course round upwards
(\(\geq 0.5\)) for final grades. A necessary reminder, as an academic, I
am not in the business of \emph{giving} out grades, I merely report the
grade that you \emph{earn}. I will not alter your grade unless you
provide a reasonable argument that I am in error (which does happen from
time to time).

\hypertarget{policies-and-expectations}{%
\section*{Policies and Expectations}\label{policies-and-expectations}}
\addcontentsline{toc}{section}{Policies and Expectations}

This syllabus is a contract between you, the student, and me, your
instructor. It has been carefully and deliberately thought out\footnote{A
  syllabus can and will be used as a legal document for disputes tried
  at a court of law. Ask me how I know.}, and I will uphold my end of
the agreement and expect you to uphold yours.

In the language of game theory, this syllabus is my commitment device. I
am a very understanding person, and I know that exceptions to rules
often need to be made for students. However, to be \emph{fair} to
\emph{all} students the syllabus artificially constrains my ability to
make exceptions at a whim for anyone. This prevents clever students from
exploiting my congenial personality at everyone else's expense. Please
read and familiarize yourself with the course policies and expectations
of you. Chances are, if you have a question, it is answered herein.

\hypertarget{online-attendance-and-participation}{%
\subsection*{Online Attendance and
Participation}\label{online-attendance-and-participation}}
\addcontentsline{toc}{subsection}{Online Attendance and Participation}

This is a hybrid course with synchronous (live) and asynchronous (on
your own time) parts.

You are generally expected to join (online via Zoom) our
\textbf{synchronous} class sessions unless circumstances prevent you
from doing so. Day-to-day attendance is not graded per se, but I
strongly recommend you join in all live sessions in which you are able,
since we all can provide live feedback and I can answer questions and
address concerns as soon as they come up. You will also benefit from a
rigid schedule and shared community.

If you are unable to make a particular class, you generally do not need
to let me know. \textbf{The videos from all class sessions are posted on
Blackboard} so please review videos of classes you were unable to attend
live.

All assignmnents are able to be completed \textbf{asynchronously} during
the week, and are \textbf{generally due by 11:59PM Sunday each week} to
allow you flexibility in your hectic schedules.

\hypertarget{late-assignments}{%
\subsection*{Late Assignments}\label{late-assignments}}
\addcontentsline{toc}{subsection}{Late Assignments}

I will accept late assignments, but will subtract a specified amount of
points as a penalty. Even if it is the last week of the semester, I
encourage you to turn in late work: some points are better than no
points!

\textbf{Homeworks}: If you turn in a homework after it is due but before
it is graded or the answer key posted, I generally will not take off any
points. However, \textbf{if you turn in a homework \emph{after} the
answer key is posted, I will automatically deduct 20 points (so the
maximum grade you can earn on it is an 80).}

\textbf{Exams}: If you know that you will be unable to complete an
\emph{exam} as scheduled for a legitimate reason, please notify me at
least \emph{one week} in advance, and we will schedule a make-up exam
date. Failure to do so, including desperate attempts to make
arrangements only \emph{after} the exam will result in a grade of 0 and
little sympathy. I reserve the right to re-weight other assignments for
students who I believe are legitimately unable to complete a particular
assignment.

\textbf{Research Project}: Starting at the deadline, I will take off 1
point for every hour that your Op-ed is late.

\hypertarget{grading}{%
\subsection*{Grading}\label{grading}}
\addcontentsline{toc}{subsection}{Grading}

I will try my best to post grades on Blackboard's Grading Center and
return graded assignments to you within about one week of you turning
them in. There will be exceptions. Where applicable, I will post answer
keys once I know most homeworks are turned in (see Late Assignments
above for penalties). Blackboard's Grading Center is the place to look
for your most up-to-date grades. See also my
\href{https://ryansafner.shinyapps.io/306_grade_calculator/}{
\texttt{Grade\ Calculator}} app where you can calculate your overall
grade using existing assignment grades and forecast ``what if''
scenarios.

\hypertarget{communication-email-slack-and-virtual-office-hours}{%
\subsubsection*{Communication: Email, Slack, and Virtual Office
Hours}\label{communication-email-slack-and-virtual-office-hours}}
\addcontentsline{toc}{subsubsection}{Communication: Email, Slack, and
Virtual Office Hours}

Students must regularly monitor their \textbf{Hood email accounts} to
receive important college information, including messages related to
this class. Email through the Blackboard system is my main method of
communicating announcements and deadlines regarding your assignments.
\textbf{Please do not reply to any automated Blackboard emails - I may
not recieve it!}. My Hood email (\texttt{safner@hood.edu}) is the best
means of contacting me. I will do my best to respond within 24 hours. If
I do not reply within 48 hours, do not take it personally, and
\emph{feel free to send a follow up email} in the very likely event that
I genuinely did not see your original message.

Our \href{https://hoodcollegeeconomics.slack.com}{slack channel} is
available to all students and faculty in Economics and Business. I have
invited all of my classes and advisees. It will not be extended to
non-Business/Economics students or faculty. All users must use their
\textbf{hood emails} and \textbf{true first and last names}. Each course
has its own channel, exclusive for verified students in the course, and
myself, by my invite only. As a third party platform, you agree to its
Terms of Service. I have created this space as a way to stay connected,
to help one another, and to foster community. Behaviors such as posting
inappropriate content, harassing others, or engaging in academic
dishonesty, to be determined solely at my discretion, will result in one
warning, the content will be deleted, and subsequent behavior will
result in a ban.

I will host general \textbf{``office hours''} on Zoom. You can join in
with video, audio, and/or chat, whichever you feel comfortable with. Of
course, if you are not available during those times, we can schedule our
own time if you prefer this method over email or Slack. If you want to
go over material from class, please have \emph{specific} questions you
want help with. I am not in the business of giving private lectures
(particularly if you missed class without a valid excuse).

Watch the excellent and accurate video
\href{https://vimeo.com/270014784}{explaining office hours} (on website
syllabus page).

\hypertarget{netiquette}{%
\subsection*{Netiquette}\label{netiquette}}
\addcontentsline{toc}{subsection}{Netiquette}

When using Zoom and Slack, please follow appropriate internet etiquette
(``Netiquette''). Written communications, like blog posts or use of the
Zoom chat, lacks important nonverbal cues (such as body language, tone
of voice, sarcasm, etc).

Above all else, please respect one another and think/reread carefully
about how others may see your post before you submit a comment. You are
expected to disagree and have different opinions, this is inherently
valuable in a discussion. Please be civil and constructive in responding
to others' comments: writing \emph{``have you considered `X'?''} is a
lot more helpful to all involved than just writing \emph{``well you're
just wrong.''}

Posting content that is wilfully incindiary, illegal, or that
constitutes academic dishonesty (such as plagarism) will automatically
earn a grade of 0 and may be elevated to other authorities on campus.

When using the chat function on Zoom or public Slack channels, please
treat it as official course communications, even though I may not be
grading it. It may be a quick and informal tool - don't feel you need to
worry about spelling or perfect grammar - but please try to avoid
\emph{too} informal ``text-speak'' (i.e.~say ``That's good for you''
instead of ``thas good 4 u'').

\hypertarget{privacy}{%
\subsection*{Privacy}\label{privacy}}
\addcontentsline{toc}{subsection}{Privacy}

\href{https://www.execvision.io/blog/maryland-call-recording-laws/}{Maryland
law}
\href{https://law.justia.com/codes/maryland/2005/gcj/10-402.html}{requires}
all parties consent for a conversation or meeting to be recorded. If you
join in, and certainly if you participate, \textbf{you are consenting to
be recorded.} However, as described below, videos are \emph{not
accessible} beyond our class.

Live lectures are recorded on Zoom and posted to Blackboard via Panopto,
a secure course management system for video. Among other nice features
(such as multiple video screens, close captioning, and time-stamped
search functions!), Panopto is authenticated via your Blackboard
credentials, ensuring that \emph{our course videos are not accessible to
the open internet.}

For the privacy of your peers, and to foster an environment of trust and
academic freedom to explore ideas, \textbf{do not record our course
lectures or discussions.} You are already getting my official copies.

The
\href{https://www2.ed.gov/policy/gen/guid/fpco/ferpa/index.html}{Family
Educational Rights and Privacy Act} prevents me from disclosing or
discussing any student information, including grades and records about
student performance. If the student is at least 18 years of age,
\emph{parents (or spouses) do not have a right to obtain this
information}, except with consent by the student.

Many of you may be tuning in remotely, living with parents, and may have
occasional interruptions due to sharing a space. This is normal and
fine, but know that I will protect your privacy and not discuss your
performance when parents (or anyone other than you, for that metter) are
present, without your explicit consent.

\hypertarget{enrollment}{%
\subsection*{Enrollment}\label{enrollment}}
\addcontentsline{toc}{subsection}{Enrollment}

Students are responsible for verifying their enrollment in this class.
The last day to add or drop this class with no penalty is
\textbf{Thursday, August 27}. Be aware of
\href{https://www.hood.edu/offices-services/registrars-office/academic-calendar}{important
dates}.

\hypertarget{honor-code}{%
\subsection*{Honor Code}\label{honor-code}}
\addcontentsline{toc}{subsection}{Honor Code}

Hood College has an Academic Honor Code which requires all members of
this community to maintain the highest standards of academic honesty and
integrity. Cheating, plagiarism, lying, and stealing are all prohibited.
All violations of the Honor Code are taken seriously, will be reported
to appropriate authority, and may result in severe penalties, including
expulsion from the college. See
\href{http://hood.smartcatalogiq.com/en/2016-2017/Catalog/The-Spirit-of-Hood/The-Academic-Honor-Code-and-Code-of-Conduct}{here}
for more detailed information.

\hypertarget{van-halen-and-mms}{%
\subsection*{Van Halen and M\&Ms}\label{van-halen-and-mms}}
\addcontentsline{toc}{subsection}{Van Halen and M\&Ms}

When you have completed reading the syllabus, email me a picture of the
band Van Halen and a picture of a bowl of M\&Ms.~If you do this
\emph{before} the date of the first exam, you will get bonus points on
the exam.\footnote{If 75-100\% of the class does this, you each get 2
  points. If 50-75\% of the class does this, you each get 4 points. If
  25-50\% of the class does this, you each get 6 points. If 0-25\% of
  the class does this, you each get 8 points.} Yes, this is real.

\hypertarget{accessibility-equity-and-accommodations}{%
\subsection*{Accessibility, Equity, and
Accommodations}\label{accessibility-equity-and-accommodations}}
\addcontentsline{toc}{subsection}{Accessibility, Equity, and
Accommodations}

College courses can, and should, be challenging and bring you out of
your comfort zone in a safe and equitable environment. If, however, you
feel at any point in the semester that certain assignments or aspects of
the course will be disproportionately uncomfortable or burdensome for
you due to any factor beyond your control, please come see me or email
me. I am a very understanding person and am happy to work out a solution
together. I reserve the right to modify and reweight assignments at my
sole discretion for students that I belive would legitimately be at a
disadvantage, through no fault of their own, to complete them as
described.

If you are unable to afford required textbooks or other resources for
any reason, come see me and we can find a solution that works for you.

This course is intended to be accessible for all students, including
those with mental, physical, or cognitive disabilities, illness,
injuries, impairments, or any other condition that tends to negatively
affect one's equal access to education. If at any point in the term, you
find yourself not able to fully access the space, content, and
experience of this course, you are welcome to contact me to discuss your
specific needs. I also encourage you to contact the
\href{https://www.hood.edu/academics/josephine-steiner-center-academic-achievement-retention/accessibility-services}{Office
of Accessibility Services} (301-696-3421). If you have a diagnosis or
history of accommodations in high school or previous postsecondary
institutions, Accessibility Services can help you document your needs
and create an accommodation plan. By making a plan through Accessibility
Services, you can ensure appropriate accommodations without disclosing
your condition or diagnosis to course instructors.

\hypertarget{tentative-schedule}{%
\section*{Tentative Schedule}\label{tentative-schedule}}
\addcontentsline{toc}{section}{Tentative Schedule}

Below is a rough sketch of the weekly schedule we will aim to follow
this semester. Each module should take approximately one class meeting.

\textbf{You can find a full schedule} with much more details, including
the readings, appendices, and other further resources for each class
meeting on the
\href{http://metricsF20.classes.ryansafner.com/schedule/}{course
website's schedule page}.

\begin{center}
\small

\begin{tabular}{llll}
\toprule
Week & Topics & Readings & Assignments\\
\midrule
8/16-8/22 & Introduction &  & \\
 & Meet R & W\&G Ch. 1 & \\
8/23-8/29 & Data Visualization with ggplot2 & W\&G Ch. 3 & \\
 & Data Wrangling with the tidyverse & W\&G Ch.5,10,11,12,18 & \\
8/30-9/5 & Optimize Workflow & W\&G Ch.8,27,28,29,30 & HW 1\\
\addlinespace
 & Data 101 \& Descriptive Statistics & Bailey A.A & \\
9/6-9/12 & Random Variables \& Distributions & Bailey A.B-I & HW 2\\
 & OLS Linear Regression & Bailey Ch.3.1, A.D-E & \\
9/13-9/19 & OLS: Goodness of Fit \& Bias & Bailey Ch.3.2-3.4,3.7-3.8 & HW 2\\
 & OLS: Precision \& Diagnostics & Bailey Ch.3 & \\
\addlinespace
9/20-9/26 & Inference for Regression' & Bailey Ch.4 & HW 3, Midterm Exam\\
 & Causal Inference & TBD & \\
9/27-10/3 & Omitted Variable Bias & Bailey Ch.5.1 & \\
 & Multivariate OLS Estimators: Bias, Precision, \& Fit & Bailey Ch.5.1,5.2,5.4 & HW 4\\
10/4-10/10 & Model Specification & TBD & \\
\addlinespace
 & Regression with Categorical Data & Bailey Ch.6.1-6.2 & \\
10/11-10/17 & Regression with Interaction Effects & Bailey Ch.6.3-6.4 & \\
 & Polynomial Regression & Bailey Ch.7.1 & \\
10/18-10/24 & Logarithmic Regression & Bailey Ch.7.2-7.4 & HW 5\\
 & Panel Data and Fixed Effects Models & Bailey Ch.8.1-8.4 & \\
\addlinespace
10/25-10/31 & Difference-in-Difference Models & Bailey Ch.8.5 & HW 6\\
 & Instrumental Variables Models & TBD & \\
11/1-11/7 & Regression Discontinuity Models & TBD & \\
 & Binary Dependent Variables Models & TBD & \\
11/8-11/14 & Classification \& Machine Learning & TBD & \\
\addlinespace
 & TBD & TBD & \\
11/15-11/21 & TBD & TBD & \\
 & TBD & TBD & \\
11/22-11/28 & Review & TBD & \\
 & Review & TBD & Final Exam\\
\bottomrule
\end{tabular}
\end{center}

\end{document}